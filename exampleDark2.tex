\documentclass{beamer}
\usetheme{PegasusUCFdark}

%%%%?!?!?!?
\usepackage{tcolorbox}
%%%%?!?!?!?

%-------------------------------
\title[GoKnights]{Interesting dataset (DarkTheme)}
\author{Alexander V. Mantzaris}
%>>>>UNCOMMENT 1 FOR YOUR DEPTARTMENT
\newcommand{\deptname}{1}
%\newcommand{\deptname}{1}%STATISTICS
%\newcommand{\deptname}{2}%MATHEMATICS
%\newcommand{\deptname}{3}%ENGINEERING AND COMPUTER SCIENCE
%\newcommand{\deptname}{4}%ELECTRICAL ENGINEERING AND COMPUTER ENGINEERING
%\newcommand{\deptname}{5}%IST INSTITUTE FOR SIMULATION AND TRAINING
%\newcommand{\deptname}{6}%PHYSICS
%\newcommand{\deptname}{7}%MECHANICAL AND AEROSPACE ENGINEERING
%\newcommand{\deptname}{8}%COMPLEX ADAPTIVE SYSTEMS LABORATORY
%\newcommand{\deptname}{9}%CENTER FOR ADVANCED TRANSPORTATION SYSTEMS SIMULATION
%\newcommand{\deptname}{10}%CHEMISTRY
%\newcommand{\deptname}{11}%BIOLOGY
%\newcommand{\deptname}{12}%ECONOMICS
%\newcommand{\deptname}{13}%SOCIOLOGY
%\newcommand{\deptname}{14}%PSYCHOLOGY
%\newcommand{\deptname}{15}%BUSINESS AND ADMINISTRATION
%\newcommand{\deptname}{16}%COLLEGE OF SCIENCES

%<<<<<
\date{\vspace{-8cm}}

%--------------------------------



%%%%%%%%%%%%%%%%
\begin{document}
%%%%%%%%%%%%%%%
\begin{frame}
  \titlepage
\end{frame}

\begin{frame}{Overview}
\tableofcontents
\end{frame}

\section{First 5 minutes}

\begin{frame}{Bayes Factors-good to know!} 
  Something different from the traditional hypothesis tests
  \begin{block}{A cool block}
    Some text
    \begin{itemize}
    \item item1
    \item item2
    \end{itemize}
  \end{block}
  More text, which can be interesting. Here we can discuss things that are good to know but not the main focus of what we want to say. Almost like optional readings. Our world is self similar so optional reading like book chapters for a course can exist in a slide as well.
\end{frame}


\subsection{A note of a long subsection}
\begin{frame}{Go Knights!}
  \vspace{-.2cm}
  Ok we are rocking\\
  And it is time for the University of Centrla Florida to have a 'dark' themed Beamer template.
  \begin{block}{block normal}
    \begin{itemize}
    \item $Y = X + E$
    \item \emph{italics}
      \item \textbf{BOLD}
    \end{itemize}
  \end{block}
  \begin{block}{We can see more theorems here}
    $1+1 = 2.00$
    \begin{itemize}
    \item filling up space
    \end{itemize}
    nice? We can see the style of the color choices for this new \emph{dark} UCF Pegasus theme. Hopefully it is easy on the eyes.
  \end{block}
  
\end{frame}

\section{second section of nothing}
\begin{frame}{Title of maybe this looks interesting for the audience}

  \begin{block}{We can see more theorems here}
    The text is interesting
  \end{block}
  \begin{block}{We can see more theorems here}
    $1+1 = 4$
    \begin{itemize}
    \item filling up space
    \item filling up space
    \item filling up space
    \item filling up space
    \item filling up space
    \item filling up space
    \end{itemize}
    OK!
  \end{block}
  
\end{frame}

\subsection{standardblock styles}

\begin{frame}{}

  \emph{{\large No title in frame}; looking at the 3 different standard blocks here}
  
  \begin{exampleblock}{example block}
    The student union is composed of brick.\\
    $\bullet$ Take a look at red brick buildings\\
    $x^2 + y^2 = r$, is important as well
  \end{exampleblock}
  
  \begin{alertblock}{alert block}    
    Simmons Hall $\not=$ Simmons Dormitory.
    \\
    Really pay attention to this$\ldots$
  \end{alertblock}

  \begin{block}{normal block}
    Some less important text\\
    now new line
  \end{block}
  
  
\end{frame}

\section{new block styles for dark theme}

\begin{frame}{introducing 2 \textbf{\emph{new}} block environments}
\begin{newblock1}{my example block:  $\{newblock1\}$}
    my block new environment\\
    using tcolorbox\\
    $\sum^N_{i=1}\pi * 2 * r$
\end{newblock1}
\begin{newblock2}{my other possible example block:  $\{newblock2\}$}
    my block new environment\\
    using tcolorbox\\
    $\sum^N_{i=1}\frac{4}{3}\pi * r^3$ 
   
\end{newblock2}

\end{frame}




%%%%%%%%%%%%%%
\end{document}
